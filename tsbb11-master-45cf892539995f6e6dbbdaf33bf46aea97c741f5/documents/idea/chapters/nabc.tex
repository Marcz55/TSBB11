\section{NABC}
This project will result in a product well vested for a flora of applications. Rigorous analysis of the market is naturally difficult and is beyond the scope of this project. What follows is a simpler analysis done according by the NABC model.

\subsection{Need}
An example application is using the image based localization for the police force. Media have recently reported very positive effects from having the police force wearing a camera. This is both for evidence, and for safety from possible police brutality. In any case the images need to be sorted and tagged. Today this requires manual work. The results of this project allows for this to be done automatically. Any taken image would get its geographical location added as a tag. Just a simple database query could yield every image taken by any police in a certain location.\\


\subsection{Approach}
Today localization is usually done by gps. This is very coarse and may be unreliable. We use only an image and some data of your location. Our approach results in a very precise localization and would rarely be wrong. When considering possible evidence, errors are not acceptable. A GPS could fail to find several important images, where our method would not. Another perk of our method is that we can estimate the direction of the camera, not only its position. This a GPS could not.\\

\subsection{Benefits}
Our method would be applicable for a plethora of customers and the estimated development time would be small. If a 3D-model of the location is available, there are little other costs involved, but initial development. Hence the cost is small. The gain is plenty of manual work removed. This manual work need to be done for each query of possible evidence images.

\subsection{Competition}
What allows this approach with little tuning is CNNs. However, this is an area quickly developing. In a couple years, perhaps there will be plenty of competetors. Hence we need to act fast.\\

Other approaches, using an array of sensors and sensor fusion would allow doing this. However, it would rely on a wide range of sensors to be available, and would require a cell phone. Our approach could use the images of anybody at the location. This is a perk we have over our competition.

